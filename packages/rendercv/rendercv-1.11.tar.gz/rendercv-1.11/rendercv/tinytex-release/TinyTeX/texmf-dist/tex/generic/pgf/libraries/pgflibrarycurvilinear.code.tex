% Copyright 2019 by Till Tantau
%
% This file may be distributed and/or modified
%
% 1. under the LaTeX Project Public License and/or
% 2. under the GNU Public License.
%
% See the file doc/generic/pgf/licenses/LICENSE for more details.

\ProvidesFileRCS{pgflibrarycurvilinear.code.tex}

%
% This file defines commands for computing points in curvilinear
% coordinate systems.
%


%
% Curvilinear coordinate systems defined in terms of a Bezier curve.
%
% For the following coordinate systems, you must first provide a
% Bezier curve (using, as always, four points), which will be set for
% the current scope. When the curve is "installed" some expensive
% precomputations are done; subsequent calls to
% \pgfpointcurvilinearxxx based on this Bezier curve will be
% relatively quick.
%



% Install a Bezier curve
%
% #1 = start point
% #2 = first control point
% #3 = second control point
% #4 = end point
%
% Description:
%
% Subsequent calls to functions like
% \pgfpointcurvilinearbezierorthogonal will be relative to the curve
% installed using this command. The main job of this macro is a
% precomputation for computing length along the curve.
% All lengths along the Bezier curve are approximated using a lookup table
% of four time/length points. For this, an approximate time for length
% 1pt is computed first, and then the length for twice this time, four
% times this time, and eight times this time are approximated. A
% distance-to-time conversion is then done by a linear interpolation
% of distance-to-time for these four points. Note that all of these
% computations are not particularly precise, but a compromise trading
% speed against precision. Also note that the results will only be
% best near the start of the curve and may be far off near the end if
% that end is degenerate (second control point very near to end
% point).
%
% Example:
%
% \pgfsetcurvilinearbeziercurve
% {\pgfpoint{0mm}{10mm}}
% {\pgfpoint{5.5mm}{10mm}}
% {\pgfpoint{10mm}{5.5mm}}
% {\pgfpoint{10mm}{0mm}} % nearly a quarter circle
% \pgfpointcurvilinearbezierorthogonal{5mm}{5mm}
%   % should be 5mm along the circle, put at
%   % distance 15mm from the origin (5mm from the circle line).

\def\pgfsetcurvilinearbeziercurve#1#2#3#4{%
  \pgf@process{#1}%
  \edef\pgf@curvilinear@line@a{\noexpand\pgfqpoint{\the\pgf@x}{\the\pgf@y}}%
  \pgf@xa=-\pgf@x%
  \pgf@ya=-\pgf@y%
  \pgf@process{#2}%
  \edef\pgf@curvilinear@line@b{\noexpand\pgfqpoint{\the\pgf@x}{\the\pgf@y}}%
  \pgf@xb=-\pgf@x%
  \pgf@yb=-\pgf@y%
  \advance\pgf@x by\pgf@xa%
  \advance\pgf@y by\pgf@ya%
  \pgfmathveclen@{\pgf@sys@tonumber\pgf@x}{\pgf@sys@tonumber\pgf@y}%
  \let\pgf@curvilinear@lenab\pgfmathresult%
  \pgf@process{#3}%
  \edef\pgf@curvilinear@line@c{\noexpand\pgfqpoint{\the\pgf@x}{\the\pgf@y}}%
  \pgf@xc=-\pgf@x%
  \pgf@yc=-\pgf@y%
  \advance\pgf@x by\pgf@xb%
  \advance\pgf@y by\pgf@yb%
  \pgfmathveclen@{\pgf@sys@tonumber\pgf@x}{\pgf@sys@tonumber\pgf@y}%
  \let\pgf@curvilinear@lenbc\pgfmathresult%
  \pgf@process{#4}%
  \edef\pgf@curvilinear@line@d{\noexpand\pgfqpoint{\the\pgf@x}{\the\pgf@y}}%
  \advance\pgf@x by\pgf@xc%
  \advance\pgf@y by\pgf@yc%
  \pgfmathveclen@{\pgf@sys@tonumber\pgf@x}{\pgf@sys@tonumber\pgf@y}%
  \let\pgf@curvilinear@lencd\pgfmathresult
  %
  \pgf@x=\pgf@curvilinear@lenab pt%
  \advance\pgf@x by\pgf@curvilinear@lenbc pt%
  \advance\pgf@x by\pgf@curvilinear@lencd pt%
  \pgfmathreciprocal@{\pgf@sys@tonumber\pgf@x}%
  \pgf@curvilinear@time@a\pgfmathresult pt%
  \pgf@process{\pgfpointcurveattime{\pgf@curvilinear@time@a}{\pgf@curvilinear@line@a}{\pgf@curvilinear@line@b}{\pgf@curvilinear@line@c}{\pgf@curvilinear@line@d}}%
  \pgf@xb=-\pgf@x%
  \pgf@yb=-\pgf@y%
  \advance\pgf@x by\pgf@xa%
  \advance\pgf@y by\pgf@ya%
  \pgfmathveclen@{\pgf@sys@tonumber\pgf@x}{\pgf@sys@tonumber\pgf@y}%
  \pgf@curvilinear@length@a\pgfmathresult pt%
  \ifdim\pgf@curvilinear@length@a>1pt\relax%
    % Ok, too large, let us make this smaller
    \pgfmathdivide@{\pgf@sys@tonumber\pgf@curvilinear@time@a}{\pgf@sys@tonumber\pgf@curvilinear@length@a}%
    \pgf@curvilinear@time@a\pgfmathresult pt%
    \pgf@process{\pgfpointcurveattime{\pgf@curvilinear@time@a}{\pgf@curvilinear@line@a}{\pgf@curvilinear@line@b}{\pgf@curvilinear@line@c}{\pgf@curvilinear@line@d}}
    \pgf@xb=-\pgf@x%
    \pgf@yb=-\pgf@y%
    \advance\pgf@x by\pgf@xa%
    \advance\pgf@y by\pgf@ya%
    \pgfmathveclen@{\pgf@sys@tonumber\pgf@x}{\pgf@sys@tonumber\pgf@y}%
    \pgf@curvilinear@length@a\pgfmathresult pt%
  \fi%
  % Compute three positions:
  \pgf@process{\pgfpointcurveattime{2\pgf@curvilinear@time@a}{\pgf@curvilinear@line@a}{\pgf@curvilinear@line@b}{\pgf@curvilinear@line@c}{\pgf@curvilinear@line@d}}
  \pgf@xa=-\pgf@x%
  \pgf@ya=-\pgf@y%
  \advance\pgf@x by\pgf@xb%
  \advance\pgf@y by\pgf@yb%
  \pgfmathveclen@{\pgf@sys@tonumber\pgf@x}{\pgf@sys@tonumber\pgf@y}%
  \pgf@curvilinear@length@b\pgfmathresult pt%
  \advance\pgf@curvilinear@length@b by\pgf@curvilinear@length@a%
  \pgf@process{\pgfpointcurveattime{4\pgf@curvilinear@time@a}{\pgf@curvilinear@line@a}{\pgf@curvilinear@line@b}{\pgf@curvilinear@line@c}{\pgf@curvilinear@line@d}}
  \pgf@xb=-\pgf@x%
  \pgf@yb=-\pgf@y%
  \advance\pgf@x by\pgf@xa%
  \advance\pgf@y by\pgf@ya%
  \pgfmathveclen@{\pgf@sys@tonumber\pgf@x}{\pgf@sys@tonumber\pgf@y}%
  \pgf@curvilinear@length@c\pgfmathresult pt%
  \advance\pgf@curvilinear@length@c by\pgf@curvilinear@length@b%
  \pgf@process{\pgfpointcurveattime{8\pgf@curvilinear@time@a}{\pgf@curvilinear@line@a}{\pgf@curvilinear@line@b}{\pgf@curvilinear@line@c}{\pgf@curvilinear@line@d}}
  \advance\pgf@x by\pgf@xb%
  \advance\pgf@y by\pgf@yb%
  \pgfmathveclen@{\pgf@sys@tonumber\pgf@x}{\pgf@sys@tonumber\pgf@y}%
  \pgf@curvilinear@length@d\pgfmathresult pt%
  \advance\pgf@curvilinear@length@d by\pgf@curvilinear@length@c%
  \let\pgf@curvilinear@comp@a\pgf@curvilinear@comp@a@initial%
  \let\pgf@curvilinear@comp@b\pgf@curvilinear@comp@b@initial%
  \let\pgf@curvilinear@comp@c\pgf@curvilinear@comp@c@initial%
  \let\pgf@curvilinear@comp@d\pgf@curvilinear@comp@d@initial%
  \let\pgf@curvilinear@comp@e\pgf@curvilinear@comp@e@initial%
  \let\pgf@curvilinear@point\pgf@curvilinear@curve@point%
}%
\newdimen\pgf@curvilinear@time@a
\newdimen\pgf@curvilinear@length@a
\newdimen\pgf@curvilinear@length@b
\newdimen\pgf@curvilinear@length@c
\newdimen\pgf@curvilinear@length@d


\def\pgf@curvilinear@comp@a@initial{%
  \pgfmathdivide@{\pgf@sys@tonumber\pgf@curvilinear@time@a}{\pgf@sys@tonumber\pgf@curvilinear@length@a}%
  \let\pgf@curvilinear@quot@a\pgfmathresult%
  \let\pgf@curvilinear@comp@a\pgf@curvilinear@comp@a@cont%
  \pgf@curvilinear@comp@a@cont%
}%
\def\pgf@curvilinear@comp@a@cont{%
  \pgf@x\pgf@curvilinear@quot@a\pgf@x%
}%

\def\pgf@curvilinear@comp@b@initial{%
  \pgf@y=\pgf@curvilinear@length@b%
  \advance\pgf@y by-\pgf@curvilinear@length@a%
  \pgfmathdivide@{\pgf@sys@tonumber\pgf@curvilinear@time@a}{\pgf@sys@tonumber\pgf@y}%
  \let\pgf@curvilinear@quot@b\pgfmathresult%
  \pgf@y\pgfmathresult\pgf@curvilinear@length@a%
  \pgf@y-\pgf@y%
  \advance\pgf@y by\pgf@curvilinear@time@a%
  \edef\pgf@curvilinear@correct@b{\pgf@sys@tonumber\pgf@y}%
  \let\pgf@curvilinear@comp@b\pgf@curvilinear@comp@b@cont%
  \pgf@curvilinear@comp@b@cont%
}%
\def\pgf@curvilinear@comp@b@cont{%
  \pgf@x\pgf@curvilinear@quot@b\pgf@x%
  \advance\pgf@x by\pgf@curvilinear@correct@b pt%
}%

\def\pgf@curvilinear@comp@c@initial{%
  \pgf@y=\pgf@curvilinear@length@c%
  \advance\pgf@y by-\pgf@curvilinear@length@b%
  \pgf@y.5\pgf@y%
  \pgfmathdivide@{\pgf@sys@tonumber\pgf@curvilinear@time@a}{\pgf@sys@tonumber\pgf@y}%
  \let\pgf@curvilinear@quot@c\pgfmathresult%
  \pgf@y\pgf@curvilinear@quot@c\pgf@curvilinear@length@b%
  \pgf@y-\pgf@y%
  \advance\pgf@y by2\pgf@curvilinear@time@a%
  \edef\pgf@curvilinear@correct@c{\pgf@sys@tonumber\pgf@y}%
  \let\pgf@curvilinear@comp@c\pgf@curvilinear@comp@c@cont%
  \pgf@curvilinear@comp@c@cont%
}%
\def\pgf@curvilinear@comp@c@cont{%
  \pgf@x\pgf@curvilinear@quot@c\pgf@x%
  \advance\pgf@x by\pgf@curvilinear@correct@c pt%
}%

\def\pgf@curvilinear@comp@d@initial{%
  \pgf@y=\pgf@curvilinear@length@d%
  \advance\pgf@y by-\pgf@curvilinear@length@c%
  \pgf@y.25\pgf@y%
  \pgfmathdivide@{\pgf@sys@tonumber\pgf@curvilinear@time@a}{\pgf@sys@tonumber\pgf@y}%
  \let\pgf@curvilinear@quot@d\pgfmathresult%
  \pgf@y\pgf@curvilinear@quot@d\pgf@curvilinear@length@c%
  \pgf@y-\pgf@y%
  \advance\pgf@y by4\pgf@curvilinear@time@a%
  \edef\pgf@curvilinear@correct@d{\pgf@sys@tonumber\pgf@y}%
  \let\pgf@curvilinear@comp@d\pgf@curvilinear@comp@d@cont%
  \pgf@curvilinear@comp@d@cont%
}%
\def\pgf@curvilinear@comp@d@cont{%
  \pgf@x\pgf@curvilinear@quot@d\pgf@x%
  \advance\pgf@x by\pgf@curvilinear@correct@d pt%
}%

\def\pgf@curvilinear@comp@e@initial{%
  \pgfmathmultiply@{8}{\pgf@sys@tonumber\pgf@curvilinear@time@a}%
  \expandafter\pgfmathdivide@\expandafter{\pgfmathresult}{\pgf@sys@tonumber\pgf@curvilinear@length@d}%
  \let\pgf@curvilinear@quot@e\pgfmathresult%
  \let\pgf@curvilinear@comp@e\pgf@curvilinear@comp@e@cont%
  \pgf@curvilinear@comp@e@cont%
}%
\def\pgf@curvilinear@comp@e@cont{%
  \pgf@x\pgf@curvilinear@quot@e\pgf@x%
}%


% Convert a distance into a time
%
% #1 = a distance
%
% Description:
%
% After having called \pgfsetcurvilinearbeziercurve, you can use this
% macro to convert a distance into a time along the curve set in that
% command. The result will be stored in \pgf@x. It will only be
% reasonably precise for small nonnegative #1 (in particular, #1
% should not be more than about half the length of the curve).

\def\pgfcurvilineardistancetotime#1{%
  \pgfmathsetlength{\pgf@x}{#1}%
  \ifdim\pgf@x<\pgf@curvilinear@length@c\relax%
    \ifdim\pgf@x<\pgf@curvilinear@length@a\relax%
      \pgf@curvilinear@comp@a%
    \else\ifdim\pgf@x<\pgf@curvilinear@length@b\relax%
      \pgf@curvilinear@comp@b%
    \else%
      \pgf@curvilinear@comp@c%
    \fi\fi%
  \else\ifdim\pgf@x<\pgf@curvilinear@length@d\relax%
    \pgf@curvilinear@comp@d%
  \else%
    \pgf@curvilinear@comp@e%
  \fi\fi%
}%



% Compute a "Bezier-orthogonal" point for use in a nonlinear transformation.
%
% #1 = x
% #2 = y
%
% Description:
%
% In this coordinate system, the x-axis "goes along" the Bezier curve
% installed using \pgfsetcurvilinearbeziercurve
% and the y-axis is always perpendicular to the (Bezier) curve at the
% given x-coordinate. Formally, given a pair (x,y), let B(x) be the point on
% the Bezier curve B at distance x from the start of the curve. Let
% T(x) be the tangent of B at B(x) and let P(x) be the (normalized)
% vector perpendicular to T(x). Then (x,y) would be mapped to B(x) +
% y*P(x). As an example, if B is a circle, then the corresponding
% curvilinear coordinate system is (essentially, except for an offset in
% the y value) the polar coordinate system.
%
% In addition to setting \pgf@x and \pgf@y, \pgf@xa/ya will be set to
% a tangent along the curve at the given point and \pgf@xb/yb to a
% tangent orthogonal to the curve.

\def\pgfpointcurvilinearbezierorthogonal#1#2{%
  \pgfmathsetmacro\pgf@curvilinear@yfactor{#2}%
  \pgfcurvilineardistancetotime{#1}%
  \pgfpointcurveattime{\pgf@x}{\pgf@curvilinear@line@a}{\pgf@curvilinear@line@b}{\pgf@curvilinear@line@c}{\pgf@curvilinear@line@d}
  \pgf@xc\pgf@x% save
  \pgf@yc\pgf@y% save
  % compute normal:
  \advance\pgf@xb by-\pgf@xa%
  \advance\pgf@yb by-\pgf@ya%
  \ifdim\pgf@xb<0.0001pt\ifdim\pgf@xb>-0.0001pt\ifdim\pgf@yb<0.0001pt\ifdim\pgf@yb>-0.0001pt\pgf@diff@curvi@ac\fi\fi\fi\fi
  \pgf@process{\pgfpointnormalised{\pgf@x=\pgf@yb\pgf@y=-\pgf@xb}}
  \pgf@x\pgf@curvilinear@yfactor\pgf@x%
  \pgf@y\pgf@curvilinear@yfactor\pgf@y%
  \advance\pgf@x by\pgf@xc%
  \advance\pgf@y by\pgf@yc%
}%

\def\pgf@diff@curvi@ac{%
  \pgf@curvilinear@line@a%
  \pgf@xa\pgf@x\pgf@ya\pgf@y%
  \pgf@curvilinear@line@c%
  \pgf@xb\pgf@x\pgf@yb\pgf@y%
  \advance\pgf@xb by-\pgf@xa%
  \advance\pgf@yb by-\pgf@ya%
  \ifdim\pgf@xb<0.0001pt\ifdim\pgf@xb>-0.0001pt\ifdim\pgf@yb<0.0001pt\ifdim\pgf@yb>-0.0001pt% still degenerate!
    \pgf@curvilinear@line@d%
    \pgf@xb\pgf@x\pgf@yb\pgf@y%
    \advance\pgf@xb by-\pgf@xa%
    \advance\pgf@yb by-\pgf@ya%
  \fi\fi\fi\fi%
  \pgf@xb-\pgf@xb%
  \pgf@yb-\pgf@yb%
}%



% Compute a "Bezier-polar" point.
%
% #1 = x
% #2 = y
%
% Description:
%
% Let (r:d) be the point (x,y) in polar coordinates, that is, r is the
% angle and d the distance of (x,y) to the origin. The point returned
% by \pgfpointcurvilinearbezierpolar is now defined as follows: First,
% we compute that point at distance d along the Bezier curve B. Let
% B(d) be this point. Then, we rotate this point around the start of
% the curve (B(0)) by r degrees.
%
% As an example, consider a triangle with one tip at the origin and
% the other tips as (4cm,3cm) and (4cm,-3cm). Then this triangle would be
% transformed as follows: We take the first 5cm of the Bezier curve
% and rotate it by roughly 37 degrees to the left and by 37 degrees to
% the right.
%
% Note that this command is pretty expensive.

\def\pgfpointcurvilinearbezierpolar#1#2{%
  \pgfmathsetlength\pgfutil@tempdima{#1}%
  \pgfmathsetlength\pgfutil@tempdimb{#2}%
  % Compute angle:
  \pgfpointnormalised{\pgfqpoint{\pgfutil@tempdima}{\pgfutil@tempdimb}}%
  \ifdim\pgfutil@tempdimb=0pt\relax\ifdim\pgfutil@tempdima=0pt\pgf@x1pt\pgf@y0pt\fi\fi%
  \ifdim\pgfutil@tempdima<0pt%
    \pgf@x-\pgf@x%
    \pgf@y-\pgf@y%
  \fi%
  \pgf@ya=-\pgf@y%
  \pgfsettransformentries%
    {\pgf@sys@tonumber\pgf@x}{\pgf@sys@tonumber\pgf@y}%
    {\pgf@sys@tonumber\pgf@ya}{\pgf@sys@tonumber\pgf@x}{0pt}{0pt}%
  \pgftransformshift{\pgfpointscale{-1}{\pgf@curvilinear@line@a}}%
  \pgfmathveclen@{\pgf@sys@tonumber\pgfutil@tempdima}{\pgf@sys@tonumber\pgfutil@tempdimb}%
  \ifdim\pgfutil@tempdima<0pt%
    \edef\pgfmathresult{-\pgfmathresult}%
  \fi%
  \pgfcurvilineardistancetotime{\pgfmathresult}%
  % Now, transform:
  \pgf@process{%
    \pgfpointadd{%
      \pgfpointtransformed{%
        \pgfpointcurveattime%
          {\pgf@x}%
          {\pgf@curvilinear@line@a}%
          {\pgf@curvilinear@line@b}%
          {\pgf@curvilinear@line@c}%
          {\pgf@curvilinear@line@d}%
      }%
    }%
    \pgf@curvilinear@line@a%
  }%
}%


\endinput
